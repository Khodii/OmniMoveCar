\chapter{Anforderungen}
Für den ersten Meilenstein, dessen Abgabe am 12.05.2019 war eine Anforderungsliste und ein Lastenheft abzugeben. Eine Liste mit 18 Anforderungen wurde noch am gleichen Tag erstellt und im Laufe der nächsten Woche ausformuliert.

Die Entscheidung, die Abgabefristen und Termine in das Lastenheft aufzunehmen, wurde getroffen da sie elementar für das Bestehen des Projektes sind, auch wenn sie schlussendlich nicht relevant für das Fahrzeug sind. 

\section{Anforderungsliste}
\begin{table}[ht]
    \resizebox{\textwidth}{!}{\begin{tabular}{|c|c|c|c|c|}
        \hline
        \multicolumn{2}{|c|}{w4 > w3 > w2 > w1} & {\large\textbf{Anforderungsliste}} &  \multicolumn{2}{|c|}{Erstellt am 7.5.}  \\ 
        \hline 
        Lfd. & F/W & Anforderung & Änderung & Verantwortlich \\ 
        \hline 
    1	& F	    & Fahren										& & Marcel       \\
    2	& F	    & Lenken können									& & Jonas        \\
    3	& F	    & Mindestdistanz 4 Meter						& & Johannes     \\
    4	& F	    & Ei transportieren können						& & Niklas       \\
    5	& F	    & Vorpräsentation: 28.5.						& & Marcel       \\
    6	& F	    & Finales Konzept und CAD Modell: 12.6.			& & Jonas        \\
    7	& F	    & Abgabetermin: 23.7.							& & Johannes     \\
    8	& F	    & Abgabe Bericht: 30.7.							& & Niklas       \\ \hline
    9	& W4	& Ei unbeschädigt transportieren				& & Marcel       \\ 
    10	& W3	& Leichtbauweise								& & Jonas        \\ 
    11	& W3	& Ressourceneffizienz							& & Johannes     \\ 
    12	& W3	& Kosteneffizienz								& & Niklas       \\ 
    13	& W3	& innovatives Design							& & Marcel       \\ \hline
    14	& W2	& Wartbarkeit									& & Jonas        \\ 
    15	& W2	& externe Steuerung oder autonomes Fahren		& & Johannes     \\ \hline
    16	& W1	& schnelles Fahren								& & Niklas       \\ 
    17	& W1	& fahren durch unwegsames Gelände				& & Marcel       \\ 
    19	& W1	& Modularisierung/Erweiterbarkeit				& & Johannes     \\ \hline
    \end{tabular}}
    \caption{Anforderungsliste}
\end{table}



\section{Lastenheft}
\subsection*{Fahren (F)}
Das finale Fahrzeug muss in der Lage sein, von Startpunkt 1 zu einem definierten Endpunkt 2 fahren zu können. Die Art und Weise der Fortbewegung steht dabei nicht im Mittelpunkt, lediglich die fehlerfreie Funktionalität muss gegeben sein.

\subsection*{Lenken können (F)}
Da Fahren allein nicht ausreicht, um die S-förmige Strecke zu absolvieren, muss das Fahrzeug ebenfalls lenkbar sein. Hierfür ist es nötig mit ausreichender Genauigkeit lenken zu können, um nicht den Bereich der Strecke zu verlassen.

\subsection*{Mindestdistanz 4 Meter (F)}
Das Fahrzeug muss in der Lage sein, beladen mit einem Ei, eine Distanz von mindestens 4 Metern zu überwinden, damit das Ei es auch bis zum Zielpunkt schafft und nicht auf halber Strecke stehen bleibt.

\subsection*{Ei transportieren können (F)}
Es wird gefordert, dass das finale Fahrzeug in der Lage ist, ein Ei über eine gewisse Distanz transportieren zu können.

\subsection*{Ei unbeschädigt transportieren (W4)}
Bei dieser Anforderung handelt es sich um eine optionale Anforderung, welche jedoch eine hohe Priorität erhalten hat (W4). Das Ei sollte also am Ende des Projekts unbeschädigt von Startpunkt 1 zu Endpunkt 2 transportiert werden können, um diese Anforderung zu erfüllen.

\subsection*{Kosteneffizienz (W3)}
Da das Budget dieses Projektes begrenzt ist, ist darauf zu achten, dass das Fahrzeug sehr kosteneffizient konstruiert und produziert wird.

\subsection*{Ressourceneffizienz (W3)}
Aufgrund der Vermeidung zusätzlicher Kosten, der limitierten Baumaterialien und der Vermeidung eines größeren Zeitaufwandes soll das Projekt möglichst ressourceneffizient geplant und umgesetzt werden. Bei dieser Anforderung handelt es sich um eine optionale Anforderung, welche jedoch eine hohe Priorität erhalten hat (W3).

\subsection*{Leichtbauweise (W3)}
Diese Anforderung ist ebenfalls optional mit recht hoher Wichtigkeit(W3). Bei der Entwicklung und Konzipierung des Fahrzeugs sollte auf ein möglichst geringes Gewicht geachtet werden.

\subsection*{Innovatives Design (W3)}
Bei dieser Anforderung handelt es sich um eine optionale Anforderung, welche eine relativ hohe Priorität erhalten hat (W3). Zusätzlich zur grundlegenden Funktionalität des Fahrzeugs wird auch noch Wert auf das optische Design des Fahrzeugs gelegt sowie die Art und Weise der Fortbewegung bewertet. Hierbei werden insbesondere kreative und einzigartige Denkansätze wertgeschätzt.

\subsection*{Externe Steuerung oder autonomes Fahren (W2)}
Um das Fahrzeug sicher durch den Parcours zu navigieren, benötigt es entweder eine externe Steuerung (z.B. Fernbedienung + IR Empfänger, Pfeiltasten der Tastatur + WLAN-Verbindung etc.) oder es muss in der Lage sein, völlig autonom (durch Sensorik und „künstliche Intelligenz“) den Parcours zu meistern.

\subsection*{Wartbarkeit (W2)}
Bei dieser Anforderung handelt es sich um eine optionale Anforderung, welche jedoch eine relativ niedrige Priorität erhalten hat (W2). Durch Achtung auf Wartbarkeit und Reparaturfreundlichkeit des Fahrzeugs kann besser auf unvorhergesehene Fehlfunktionen reagiert werden.

\subsection*{Modularisierung/Erweiterbarkeit (W1)}
Für die Aspekte „Kosteneffizienz“, „Ressourceneffizienz“ und „Wartbarkeit“ spielt die modulare Entwicklung und Konstruktion eine große Rolle. Durch die Modularisierung kann das Fahrzeug leichter überarbeitet, erweitert und gewartet werden. Des Weiteren ist eine zukünftige Erweiterung des Projekts dadurch leichter umsetzbar. Bei dieser Anforderung handelt es sich um eine optionale Anforderung, welche eine niedrige Priorität erhalten hat (W1).

\subsection*{Schnelles Fahren (W1)}
Da in der Logistikbranche Zeitdruck ein wesentlicher Faktor ist, besteht der Wunsch, dass das Fahrzeug die zurückzulegende Strecke in möglichst kurzer Zeit bewältigt und sich deshalb mit hohem Tempo fortbewegt.

\subsection*{Fahren durch unwegsames Gelände (W1)}
Das finale Fahrzeug kann sich im Optimalfall nach dem Abschluss des Projekts auch auf unwegsamen Geländen effizient und zielsicher fortbewegen. Diese Anforderung ist nicht explizit durch den Auftraggeber vorgegeben, würde jedoch einen Mehrwert des Produkts erzielen. Daher handelt es sich um eine optionale Anforderung, welche eine niedrige Priorität erhalten hat (W1).
