\chapter{Planung}
\section{Entwürfe}
\subsection{Seggway}

\subsection{Weggchair}

\subsection{TEGGLA}

\section{Morphologischer Kasten}

\section{Online Bestellungen}
Um das Fahrzeug nach dem Praktikum behalten zu können, war eine Voraussetzung, dass nur Teile verbaut werden, die nicht Eigentum des Lehrstuhls sind.
Aus diesem wurden die nötigen Bauteile bei unterschiedlichen Onlineshops herausgesucht und bestellt.
Hierbei fiel die Entscheidung auf Pollin, einem deutschen Elektronik Händler und AliExpress, einem chinesischen Großhändler.
In der ursprünglichen Planung wurden die Kosten pro Fahrzeug auf circa \EUR{20} - \EUR{25} überschlagen.
In der finalen Bestellung beliefen sich die Kosten auf insgesamt etwa \EUR{32}.

Siehe Tabelle \ref{table:Kosten} für eine genaue Aufteilung der Kosten.
\begin{table}[!htbp]
	\centering
\begin{tabular}{|c|c|c|c|}
	\hline 
	Artikel & Stücke & \euro/Stück & Quelle \\ 
	\hline 
	\hline
	&  &  &  \\ 
	\hline
	&  &  &  \\ 
	\hline 
	&  &  &  \\ 
	\hline 
	&  &  &  \\ 
	\hline 
	&  &  &  \\ 
	\hline 
	&  &  &  \\ 
	\hline 
	&  &  &  \\ 
	\hline 
	&  &  &  \\ 
	\hline 
	&  &  &  \\ 
	\hline 
	&  &  &  \\ 
	\hline 
	&  &  &  \\ 
	\hline 
	&  &  &  \\ 
	\hline 
	&  &  &  \\ 
	\hline 
	&  &  &  \\ 
	\hline 
	&  &  &  \\ 
	\hline 
	&  &  &  \\ 
	\hline 
	&  &  &  \\ 
	\hline 
\end{tabular} 
\caption{Kostenübersicht} 
\label{table:Kosten}
\end{table} 



\section{Verwendete Technologien}

\subsection{Git}
\subsection{Creo}
\subsection{PlatformIO}