%% ++++++++++++++++++++++++++++++++++++++++++++++++++++++++++++
%% Kapitel 3: Allgemeine Hinweise
%% ++++++++++++++++++++++++++++++++++++++++++++++++++++++++++++
%
%  Ger�st:
%  * Version 0.10
%  * Dipl.-Ing. Florian Evers, florian.evers@tu-ilmenau.de
%  * Fachgebiet Kommunikationsnetze, TU Ilmenau
%
%  F�r Hauptseminare, Studienarbeiten, Diplomarbeiten
%
%  Autor           : Max Mustermann
%  Letzte �nderung : 31.12.2015
%

\chapter{Allgemeine Hinweise}
\section{\LaTeX-bezogen}
\begin{description}
\item[Abk�rzungsverzeichnis]
      Sollte das Abk�rzungsverzeichnis nach dem Hinzuf�gen eines
      {\ttfamily nomenclature}"=Kommandos nicht aktualisiert werden,
      muss der {\ttfamily makeindex}"=Aufruf
      manuell in der Konsole gestartet werden. Manche
      Entwicklungsumgebungen machen dies aber schon automatisch.
      Bitte die genannten Parameter nicht vergessen!

      Bei Benutzern der GUI \texttt{Kile} kann es vorkommen,
      dass der \texttt{makeindex}-Befehl nicht automatisch ausgef�hrt
      wird, scheint ein Bug zu sein. In diesem Fall kann der Index
      auch manuell durch Aufruf von \texttt{makeindex} aktualisiert
      werden.
\item[Thesenpapier]
      F�r die Thesen wurde mit der Version 0.8 an ein eigenst�ndiges
      Dokument namens {\ttfamily thesen-handout.tex} hinzugef�gt.
      Es bindet ebenso wie das Hauptdokument die Datei
      {\ttfamily thesen.tex} ein, erzeugt aber eben nur dieses
      eine Blatt ohne eine Seitenzahl.
\item[Beidseitiger Druck]
      Im Zentraldokument {\ttfamily dokument.tex} kann das Layout
      auf doppelseitigen Druck umgeschaltet werden (Option
      {\ttfamily twoside} statt {\ttfamily oneside}). Allerdings
      verlangen manche Pr�fungs�mter explizit einen einseitigen
      Druck! Neue Kapitel ({\ttfamily chapter}) beginnen dabei
      automatisch auf einer Vorderseite (\(\to \) rechte Seite).
      Die R�nder sind dabei innen nur halb so breit wie au�en, was
      aber Absicht ist: Zusammen ergeben die linke und die rechte
      Seite innen einen "`wei�en Streifen"', der genauso breit ist wie die
      �u�eren R�nder.
\item[�berlange Kapitel�berschriften]
      Manchmal m�ssen �berschriften sehr lang sein, sodass sie von \LaTeX\ 
      umgebrochen werden. Dieses Verhalten ist aber weder im Inhaltsverzeichnis
      noch in der Kopfzeile erw�nscht! Daher kann man zu einer �berlangen
      �berschrift auch eine Kurzform mit angeben, welche dann im Inhaltsverzeichnis
      und im Dokumentenkopf verwendet wird:\\
      {\ttfamily \textbackslash chapter[Kurzform]\{Langform\}}
\item[Einz�ge]
      Bitte \emph{nicht!} die Einz�ge �ndern oder abschalten. Das ist
      so gewollt und verbessert den Lesefluss! (Stichwort
      \texttt{\textbackslash setlength\textbackslash parindent\{0pt\}})!
\item[BibTeX-Eintr�ge mit mehreren Autoren]
      Sollen mehrere Autoren angegeben erden, so sind diese einzeln
      als \emph{Vorname Nachname} anzugeben und durch \texttt{and}
      voneinander zu trennen. BibTeX ersetzt das \texttt{and} dann
      durch das deutsche "`und"':\\
      \texttt{author = \{Adam Riese and Eva Zwerg\},}
\end{description}




\section{Inhaltlich}
\begin{itemize}
\item �berschriften im Inhaltsverzeichnis nie tiefer als
      vier Ebenen. Dies geht mit \LaTeX\ auch gar nicht anders,
      da {\ttfamily subsubsection} bereits die niedrigste
      Schachtelungstiefe darstellt, welche noch im
      Inhaltsverzeichnis aufgef�hrt wird.
\item Die Kapitel sollten in der sp�teren Ausarbeitung anders
      benannt werden als in dieser Formatvorlage. Eine Diplomarbeit
      \emph{kann} beispielsweise aus der folgenden Aufteilung bestehen:

      \begin{enumerate}
      \item Problemstellung
      \item Theoretische Grundlagen
      \item Herleitung
      \item Der Prototyp
      \item Zusammenfassung
      \item Ausblick
      \end{enumerate}
\item Eine Inventarisierungsnummer ist nicht bei jeder Art der
      Ausarbeitung gegeben. Dazu kann im Zentraldokument
      (\(\to \) {\ttfamily dokument.tex}) eine leere Zeichenfolge
      hinterlegt werden.
\item Es empfiehlt sich, ein Programm zur Rechtschreibpr�fung zu
      installieren. Alternativ zu einer \LaTeX"=f�higen
      Rechtschreibkorrektursoftware kann ein Abschnitt auch
      in bspw. Microsoft Word getippt und gepr�ft werden, bevor
      er dann in das \LaTeX"=Dokument eingef�gt wird.
\item F�r Diplomarbeiten wird generell ein englischer "`Abstract"'
      ben�tigt!
\end{itemize}
