\chapter{Zusammenfassung}
In diesem abschließenden Kapitel werden die gesammelten Erfahrungen reflektiert, sowie die wichtigsten erlernten, beziehungsweise verbesserten Fähigkeiten beim Entwickeln des TEGGLA zusammengefasst.


\section{Fazit}
Das Entwickeln des TEGGLA hat dem gesamten Team sehr viel Spaß gemacht und um das erwünschte Ergebnis zu erreichen, wurde der Umgang mit vielen neuen Technologien erlernt beziehungsweise das bereits vorhandene Wissen vertieft. 
Vor allem das Konstruieren in CAD, die Anfertigung von technischen Zeichnungen und das tatsächliche Drucken und Zusammenbauen der Baugruppen hatten einen großen Mehrwert für unsere zukünftigen Projekte. 
Die wichtigsten verwendeten Technologien waren:

\subsection*{Creo}
Konstruieren aller Bauteile des TEGGLA.

\subsubsection*{C++}
Programmierung des MicroController auf einer Hardwarenahen Ebene.

\subsubsection*{JavaScript / HTML5}
Weboberfläche mit Sockets für die Kommunikation und Steuerung des TEGGLA.

\subsubsection*{Github}
Repository für die Sicherung, Versionierung, Verteilung und Aktualsierung aller Daten, technischen Zeichnungen, Skizzen, Berichte, Präsentationen und Quellcode.


\section{Errungene Erfahrung}
Der wohl zeitaufwändigste und schwierigste Aspekt des Projekts war der 3D-Druck eines Planetengetriebes, welches die erforderliche Umsetzung für das omnidirektionale Fahren bereitstellen kann. Bereits geringe Druckerungenauigkeiten führten zu nicht nutzbaren Getriebeversionen und daher wurden viele Anläufe benötigt, um ein passendes Planetengetriebe zu entwickeln. Für zukünftige Projekte mit 3D Druckern wird das berücksichtigt und der Fokus bei der Entwicklung darauf gelegt.  

Als sehr positiv war die zeitliche Planung und Zusammenarbeit des Teams zu bewerten. Trotz unerwarteten Mehraufwänden für Teilaufgaben, kam es zu keinerlei zeitlichen Engpässen und dadurch zu einem qualitativ hochwertigen Endprodukt. Eine möglichst präzise Aufgabendefinition und -verteilung am Projektanfang ist enorm wichtig um dies sicherstellen zu können. Zusätzlich waren die Meilensteintermine des Lehrstuhl ebenfalls hilfreich um sich an den Terminplan zu halten.

Die Nutzung eines GIT Repositories war ebenfalls äußerst gewinnbringend, da die Versionierung, Verteilung, Aktualisierung und Sicherung jederzeit und für alle Teammitglieder gewährleistet ist. 

Insgesamt wurden enorm viele Erfahrungen über den Entwicklungsprozess eines lenkbaren Fahrzeugs gesammelt, welche sich natürlich auch auf andere Produkte und Projekte abstrahieren lassen. Die einzelnen Entwicklungsschritte, nämlich die Auswahl einer Idee, das Anfertigen eines Lastenhefts, die Rollen- und Aufgabenverteilung innerhalb des Teams sowie die zeitliche Einhaltung von Terminen sind alles wichtige Kenntnisse für unsere zukünftigen Projekte.
