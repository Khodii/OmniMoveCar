\chapter{Einleitung \& Motivation}
Im Rahmen des Praktikums für Produktionstechnik an der Universität Augsburg im Sommersemester 2019 wurden lenkbare Fahrzeuge entwickelt und durch die Verwendung von 3D Druckern realisiert. Die Aufgabe des Praktikums bestand darin, ein herkömmliches Ei mit Hilfe des selbst entwickelten Autos unbeschädigt durch einen vom Lehrstuhl definierten Parcours zu transportieren.

In diesem Bericht wird detailliert auf die Planung, Anforderungsermittlung, Systementwicklung und Systemvalidierung eingegangen. 

Zusätzlich werden die Schlüsseltechnologien des TEGGLA beschrieben und weitere interessante Teilaspekte wie mögliche Erweiterungs- und Verbesserungsmöglichkeiten diskutiert. Des weiteren befinden sich im Anhang des Berichts noch alle relevanten technischen Daten, der gesamte Quellcode sowie alle technischen Zeichnungen der verschiedenen Baugruppen.

