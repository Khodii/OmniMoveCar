%% ++++++++++++++++++++++++++++++++++++++++++++++++++++++++++++
%% Thesen zur Ausarbeitung. F�r Diplomarbeiten
%% ++++++++++++++++++++++++++++++++++++++++++++++++++++++++++++
%
%  Ger�st:
%  * Version 0.10
%  * Dipl.-Ing. Florian Evers, florian.evers@tu-ilmenau.de
%  * Fachgebiet Kommunikationsnetze, TU Ilmenau
%
%  F�r Hauptseminare, Studienarbeiten, Diplomarbeiten
%
%  Autor           : Max Mustermann
%  Letzte �nderung : 31.12.2015
%

\chapter*{Thesen zur \artderausarbeitung}
\addcontentsline{toc}{chapter}{Thesen zur \artderausarbeitung}
\ihead[]{Thesen zur \artderausarbeitung}

\begin{enumerate}
\item Mit \LaTeX\ gesetzte Dokumente sehen �berall
      gleich aus. Sie werden �hnlich wie HTML in Klartext
      geschrieben und anschlie�end mit Hilfe eines Konverters in
      Postscript- oder PDF"=Dateien gewandelt.
\item \LaTeX\ gibt es f�r alle wichtigen Betriebssysteme.
\item Die Benutzung einer integrierten Entwicklungsumgebung,
      beispielsweise {\ttfamily Kile} oder {\ttfamily TeXnicCenter},
      wird empfohlen.
\item Dieses Dokument ist Formatvorlage und Einstiegshilfe
      zugleich. Einfach den Text durch die eigene Ausarbeitung
      ersetzen.
\end{enumerate}

% Etwas Platz schaffen:
\section*{}

Ilmenau, den 31.\,12.\,2015\hfill \namedesautors
